\documentclass{article}
\usepackage{amsmath}
\usepackage{graphicx}
\usepackage{hyperref}
\usepackage{ctex}

\title{布谷鸟哈希实现与性能分析}
\author{余厚佚 3230101536}
\date{\today}

\begin{document}

    \maketitle

    \section{引言}
    布谷鸟哈希(Cuckoo Hashing)是一种高效的哈希表实现方法,能够在最坏情况下提供常数时间复杂度的查找操作。本文将介绍布谷鸟哈希的设计思路,并通过一系列测试验证其在不同装载率下的性能表现。

    \section{布谷鸟哈希设计思路}
    布谷鸟哈希的核心思想是利用多个哈希函数来减少冲突。具体来说,每个元素可以放置在由两个或更多哈希函数计算出的位置之一。当插入新元素时,如果目标位置已有其他元素,则该元素会被“踢”到它的备用位置,依此类推,直到找到空位或达到最大重哈希次数。

    \subsection{关键特性}
    1. \textbf{多哈希函数}:使用至少两个独立的哈希函数,确保每个元素有多个可能的存储位置。
    2. \textbf{固定大小的数组}:所有元素都存储在一个固定大小的数组中,避免了动态调整大小带来的额外开销。
    3. \textbf{重新哈希机制}:当发生过多冲突时,整个哈希表会进行重新哈希,以保证算法的效率和稳定性。

    \section{测试集构建思路}
    为了全面评估布谷鸟哈希的性能,构建了多个装载率下的测试数据集。每个测试集包含一定数量的随机字符串,用作插入哈希表的元素。

    \subsection{测试参数}
    \begin{itemize}
        \item \textbf{初始哈希表大小}:设置为1009个桶,这是一个素数,有助于降低冲突概率。
        \item \textbf{装载率}:选择了几个不同的装载率,包括0.1、0.2、0.3、0.4和0.45。
        \item \textbf{测试数据生成}:使用伪随机数生成器创建长度固定为10的随机字符串。
    \end{itemize}

    \subsection{测试过程}
    根据设定的装载率生成数据,逐个插入哈希表并测量插入时间。测试完成后,计算每次插入的平均耗时,并验证插入后的表中元素数量是否与期望值一致。

    \section{测试结果分析}
    实验结果表明,在装载率低于0.5时,插入时间接近常数,符合布谷鸟哈希的设计目标。以下是部分测试结果:

    \begin{table}[h!]
        \centering
        \begin{tabular}{|c|c|c|}
            \hline
            装载率 & 总插入时间 (秒) & 平均插入时间 (微秒) \\
            \hline
            0.10 & 0.00 & 0.53 \\
            0.20 & 0.00 & 0.40 \\
            0.30 & 0.00 & 1.80 \\
            0.40 & 0.00 & 0.40 \\
            0.45 & 0.00 & 1.16 \\
            \hline
        \end{tabular}
        \caption{不同装载率下的插入性能测试}
    \end{table}

    \section{性能分析与验证}
    \subsection{复杂度验证}
    通过测试不同装载率下的插入耗时,可以观察到在装载率低于0.5时,插入操作的平均耗时几乎保持恒定,验证了布谷鸟哈希在低装载率下的 $O(1)$ 性能。

    \subsection{重哈希开销}
    在装载率较高时(接近0.5),重哈希操作的开销开始显现,但总耗时仍在可接受范围内。这说明布谷鸟哈希在高装载率下依然能够保持较高的性能稳定性。

    \section{结论}
    本文通过实现布谷鸟哈希并在不同装载率下进行测试,验证了其高效性和稳定性。结果表明,在装载率低于0.5的情况下,布谷鸟哈希能够实现接近常数时间的插入操作。同时,通过合理设计哈希函数和选择表大小,可以进一步优化其性能。

\end{document}
