\documentclass[UTF8]{ctexart}
\usepackage{geometry}
\geometry{margin=1.5cm, vmargin={0pt,1cm}}
\setlength{\topmargin}{-1cm}
\setlength{\paperheight}{29.7cm}
\setlength{\textheight}{25.3cm}

% useful packages.
\usepackage{amsfonts}
\usepackage{amsmath}
\usepackage{amssymb}
\usepackage{amsthm}
\usepackage{enumitem} % 使用 enumitem 包代替 enumerate
\usepackage{graphicx}
\usepackage{multicol}
\usepackage{fancyhdr}
\usepackage{layout}
\usepackage{listings}
\usepackage{float, caption}

\lstset{
    basicstyle=\ttfamily, basewidth=0.5em
}

% some common command
\newcommand{\dif}{\mathrm{d}}
\newcommand{\avg}[1]{\left\langle #1 \right\rangle}
\newcommand{\difFrac}[2]{\frac{\dif #1}{\dif #2}}
\newcommand{\pdfFrac}[2]{\frac{\partial #1}{\partial #2}}
\newcommand{\OFL}{\mathrm{OFL}}
\newcommand{\UFL}{\mathrm{UFL}}
\newcommand{\fl}{\mathrm{fl}}
\newcommand{\op}{\odot}
\newcommand{\Eabs}{E_{\mathrm{abs}}}
\newcommand{\Erel}{E_{\mathrm{rel}}}

\begin{document}

\pagestyle{fancy}
\fancyhead{}
\lhead{余厚佚, 3230101536}
\chead{数据结构与算法第5次作业}
\rhead{Nov.4th, 2024}

\section{代码实现说明}
\section{标准的二叉搜索树删除过程}
函数的实现首先遵循二叉搜索树的删除规则,根据节点情况分为三种情况进行处理:
   叶节点删除:如果目标节点没有子节点,直接删除。
   单子节点删除: 如果节点只有一个子节点,则用该子节点替代当前节点。
   双子节点删除: 如果节点有两个子节点,则用右子树中的最小节点(即中序后继)替代该节点的值,并递归删除最小节点。这一过程保证了树的有序性。

\section{更新节点高度}
删除操作完成后,沿路径回溯至根节点,更新所有祖先节点的高度。节点的高度定义为 $1 + \max(\text{左子树高度}, \text{右子树高度})$,因此在子树发生变化时,需要重新计算其祖先节点的高度。
\section{重新平衡 AVL 树}
更新高度后,可能导致某些节点的平衡性被打破。此时,检查每个节点的平衡因子(即左右子树高度差)是否超过 1。如果不平衡,则根据具体的情况进行相应的旋转操作,以恢复树的平衡性。这些旋转操作包括:
    单右旋转:当节点的左子树的左侧过高时,执行右旋转。
    单左旋转:当节点的右子树的右侧过高时,执行左旋转。
    双旋转:左-右旋转:当节点的左子树的右侧过高时,先对左子节点进行左旋转,再对当前节点进行右旋转。
    双旋转:右-左旋转:当节点的右子树的左侧过高时,先对右子节点进行右旋转,再对当前节点进行左旋转。

\section{结论}

通过实现上述步骤的remove函数,可以在 AVL 树中删除节点并保持平衡。

\end{document}

%%% Local Variables:
%%% mode: latex
%%% TeX-master: t
%%% End: