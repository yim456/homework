\documentclass{article}
\usepackage{ctex}  % 使用 ctex 宏包支持中文
\usepackage{amsmath}
\usepackage{graphicx}
\usepackage{booktabs}
\usepackage{listings}  % 用于代码块
\usepackage{xcolor}    % 用于代码高亮

% 设置代码块样式
\lstset{
    basicstyle=\ttfamily,
    keywordstyle=\color{blue},
    commentstyle=\color{green},
    stringstyle=\color{red},
    showstringspaces=false,
    breaklines=true,
    numbers=left,
    numberstyle=\tiny,
    frame=single,
    rulecolor=\color{gray}
}

\title{数据结构与算法第七次作业}
\author{余厚佚 3230101536}
\date{\today}

\begin{document}

\maketitle

\section{整体设计思路}

    \begin{enumerate}
        \item HeapSort.h:
        \begin{itemize}
            \item heapify 函数:通过递归方式调整子树,从而确保该节点的子树满足最大堆的性质。
            \item heapSort 函数:构建最大堆,反复交换堆顶元素与末尾元素,实现排序的目的。
        \end{itemize}
        \item test.cpp:
        \begin{itemize}
            \item 按照评分标准要求,生成四种不同类型的序列进行测试:随机序列、有序序列、逆序序列和部分重复元素序列,每种序列的长度均不小于1000000。
            \item 记录两种堆排序在不同序列的排序时间,进行比较。(以及通过check函数验证排序的正确性)
        \end{itemize}
    \end{enumerate}

    \section{主要函数的功能和实现细节}

    \subsection{heapify函数}

    \subsubsection{功能}

    heapify 函数的主要作用是维护堆的性质。对于给定的 i ,确保以 i 为根的子树满足最大堆的条件。

    \subsubsection{实现细节}

    template <typename T>
    void heapify(std::vector<T>\& arr, int n, int i)
    {
        int largest = i; // 初始化最大值为根节点
        int left = 2 * i + 1; // 左子节点索引
        int right = 2 * i + 2; // 右子节点索引

        if (left < n \&\& arr[left] > arr[largest])    // 如果左子节点存在且大于根节点,更新最大值
            largest = left;

        if (right < n \&\& arr[right] > arr[largest])  // 如果右子节点存在且大于当前最大值,更新最大值
            largest = right;

        if (largest != i)                            // 如果最大值不是根节点,交换并递归
        {
            std::swap(arr[i], arr[largest]);
            heapify(arr, n, largest);
        }
    }

    \begin{itemize}
        \item 传入的参数:
        \begin{itemize}
            \item arr::还没变成堆的数组。
            \item n:当前堆的大小。
            \item i:当前需要变成堆的索引。
        \end{itemize}
        \item 具体实现步骤:
        \begin{enumerate}
            \item 假设当前节点 i 是最大值。
            \item 比较子节点的值,找出三者中的最大值。
            \item 如果最大值不是当前节点,则交换,并递归,从而确保子树满足堆的性质。
        \end{enumerate}
    \end{itemize}

    \subsection{heapSort函数}

    \subsubsection{功能}

    heapSort 函数通过构建最大堆并反复交换堆顶元素与末尾元素,从而实现升序排序。

    \subsubsection{实现细节}

    template <typename T>
    void heapSort(std::vector<T>\& arr)
    {
        int n = arr.size();

        for (int i = n / 2 - 1; i >= 0; --i)    // 从最后一个非叶子节点向前遍历每个节点,利用heapify函数来确保子树满足最大堆的性质
            heapify(arr, n, i);

        for (int i = n - 1; i >= 0; --i)         // 通过循环,逐步将堆中的最大元素移到数组末尾
        {
            std::swap(arr[0], arr[i]);
            heapify(arr, i, 0);
        }
    }

    \begin{itemize}
        \item 说明:
        \begin{enumerate}
            \item 构建最大堆:
            \begin{itemize}
                \item 从最后一个非叶子节点开始,调用 heapify 函数,确保满足最大堆的性质。
            \end{itemize}
            \item 排序过程:
            \begin{itemize}
                \item 将堆顶元素(当前最大值)与数组末尾元素交换。
                \item 缩小堆的有效范围(即忽略已排序的末尾元素)。
                \item 调用 heapify 重新调整堆,确保新的堆顶元素是当前堆的最大值。
            \end{itemize}
        \end{enumerate}
    \end{itemize}

    \section{测试流程}

    \subsection{生成要求的四种测试序列}

    \subsection{具体测试步骤}

    \begin{enumerate}
        \item 打印排序前序列的前50个元素:得到序列的初始状态,便于起见,只输出前50个元素。

        \item 自定义堆排序:
        \begin{itemize}
            \item 记录开始时间。
            \item 调用我写的 heapSort 函数进行排序。
            \item 记录结束时间,计算排序耗时。
            \item 验证排序结果的正确性(利用check函数)。
            \item 打印排序后的前50个元素。
        \end{itemize}

        \item std::sort\_heap()排序:
        \begin{itemize}
            \item 重新生成堆(调用 std::make\_heap())。
            \item 记录开始时间。
            \item 调用 std::sort\_heap() 对序列进行排序。
            \item 记录结束时间,计算排序耗时。
            \item 验证排序结果的正确性。
            \item 打印排序后的前50个元素。
        \end{itemize}

    \end{enumerate}

    \section{测试结果与对比}

\begin{table}[h!]
\centering
\begin{tabular}{@{}lll@{}}
\toprule
序列类型 & 自定义堆排序时间 (秒) & \lstinline|std::sort_heap| 时间 (秒) \\ \midrule
随机序列 & 0.0784779 & 0.0951558 \\
有序序列 & 0.0386108 & 0.0493739 \\
逆序序列 & 0.0415211 & 0.0538083 \\
部分重复序列 & 0.0459754 & 0.0498996 \\ \bottomrule
\end{tabular}
\caption{性能比较}
\label{tab:performance_comparison}
\end{table}

\section{时间复杂度分析及效率差异原因}

\subsection{时间复杂度分析}

堆排序主要由以下两个过程组成:

\begin{enumerate}
    \item 构建最大堆:
    \begin{itemize}
        \item 从最后一个非叶子节点开始,依次调用 heapify 函数,确保整个数组满足最大堆的性质。
        \item 每次 heapify 的时间复杂度为 $O(\log n)$。
        \item 总体构建堆的时间复杂度为 $O(n)$。
    \end{itemize}

    \item 排序过程:
    \begin{itemize}
        \item 进行 $n$ 次交换,将堆顶元素(当前最大值)移到数组末尾。
        \item 每次交换后,调用 heapify 重新调整堆,时间复杂度为 $O(\log n)$。
        \item 总体排序过程的时间复杂度为 $O(n \log n)$。
    \end{itemize}
\end{enumerate}

\subsection{自定义堆排序与std::sort\_heap()效率差异的原因}

在我的代码中,经过测试可以发现,实际上 std::sort\_heap() 运行的更慢QAQ。或许是因为生成的是整型数据==不过一般来说,应该是标准库的更快点,主要原因可能如下:

    \begin{enumerate}
        \item 更低级的运算符:
            \begin{itemize}
                \item 使用了更高效的位运算和内存访问模式,提高了执行效率。
            \end{itemize}

        \item 迭代实现:
        \begin{itemize}
            \item 标准库的heapify 可能采用了迭代方式实现,避免了递归调用带来的栈开销和函数调用开销。
        \end{itemize}
    \end{enumerate}

\end{document}
